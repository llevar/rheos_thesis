In this chapter we describe a software framework called Rheos, which demonstrates an approach for reasoning about large genomic datasets utilizing concepts of service-orientation and data streaming in contrast with traditional genomic data analysis frameworks\autocite{depristo2011framework} that take a procedural batch-based approach. Rheos' focus on service-orientation and streaming allow the users to make active tradeoff decisions between analysis time, cost, and quality as well as setting up precise operational Service Level Agreements, both between Rheos components, and between Rheos and external systems, as we describe in detail below.

\section{General Framework Design}

As already discussed in Chapters \ref{ch:introduction} and \ref{ch:background}, the general problem consists of collecting DNA samples from a population of individuals under study, sequencing these samples using Next Generation Sequencing techniques, identifying the mutations that are present, annotating their functional impact and utilizing the obtained data in a downstream data analysis with research or clinical decision-making goals. While there is a great variety of possible downstream analyses that may be performed depending on the individual goals of the analyst, there is a fairly well established set of steps for processing of the raw NGS data into a set of annotated variants, and it is these steps that we target with this work. The typical approach that is in widespread use today is to collect a batch of samples and then process each sample individually with a sequence of individual tools, that may be described via a higher-level workflow construct (such as in Figure \ref{fig:gatk_best_practices}, or using a framework like Butler, as described in Chapters \ref{ch:butler_architecture}, \ref{ch:butler_implementation}). There are, however, a number of factors that leave room for improvement in this model. These improvements lie along a set of dimensions that we describe briefly in the Introduction via a utility function $U_i = C_i + T_i + A_i$ for sample $i \in [1,N_s]$  that needs to be optimized, and that we describe in more detail here.

We use the following definitions throughout the text:

\begin{table}[!ht]
    \caption{Rheos common definitions}
    \label{tab:rheos_notation}
    {\begin{tabular}{lp{7cm}}
    \toprule
    Symbol & Description \\
    \midrule
    $N_p$ & Number of people \\
    $P = \{p_i : i \in [1,N_p]\}$ & Set of individuals under study \\
    $N_s$ & Number of samples \\
    $S = \{s_i : i \in [1,N_s]\}$ & Set of sequenced DNA samples. Each individual can have one or more samples. \\
    $A_i $ & Accuracy score of analysis for sample $i$ (precise definition of Accuracy TBD) \\
    $C_i = c_{g_i} + c_{s_i} + c_{a_i} + c_{r_i}$ & Cost score of data generation, storage, analysis, and retrieval respectively \\
    $T_i$ & Time score to process sample $i$ \\
    $U_i = C_i + T_i + A_i$ & A utility function for individual $i$ that penalizes high cost, high processing time, and low accuracy\\
    $U = \sum_{i=1}^{N_s} U_i$ & Overall utility of processing $N_s$ samples through Rheos.\\
    \bottomrule
    \end{tabular}}
\end{table}


\section{Data Streaming Architecture}

Set up the general architecture of information flow through the system - types of entities, communication channels, contracts, SLAs, capabilities:

\begin{itemize}
 \item Each stream is a collection of randomly ordered datagrams.
 \item A datagram may constitute data or an event.
 \item The rate of the stream is understood.
 \item A service applies a transformation to the stream by virtue of making a decision upon inspecting stream contents.
 \item A service makes decisions based on:
 \begin{itemize}
    \item Individual datagrams
    \item Sliding windows of datagrams
    \item Particular sequences of datagrams
    \item Integration of incoming stream signal with a static data store
    \item Combinations of multiple streams
 \end{itemize}
 \item A service emits a new stream as output.
 \item A service may decorate an existing stream with new info or create new datagrams.
 \item A service may persist data to a permanent store.
 \item The rate at which the service can make decisions is understood (along with limiting conditions).
 \item Communication between services is mediated via queues.
 \item Services provide an SLA on uptime, latency, throughput, etc.
 \item Services fail fast.
 \item System output supports push and pull modes.
\end{itemize}
\section{Domain-specific Problems}

Describe specific problems that need to be addressed by the system in a stream-based formulation - how to go from a collection of raw reads to a set of annotated variants: map reads, perform QC filtering, model loci, assemble alternative haplotypes, call variants, filter variants, annotate, produce output.  

\begin{itemize}
    \item Perform read QC
    \item Align reads to reference
    \item Collect read stream statistics
    \item Assemble local read contigs
    \item Model individual loci
    \item Call variants (maybe need to split by variant types)
    \item Genotype variants
    \item Filter variants
    \item Annotate variants
    \item Output variants
\end{itemize}


\section{Services of Rheos}

Provide a mapping from the domain-specific problems onto a particular implementation in the data streaming architecture. List services, their responsibilities, contracts, etc.

\begin{description}
    \item [Metadata] - Take in metadata related to patients, samples, files, etc. In: Metdata records, Out: ingestion confirmation events.
    \item [Read Streaming] - Take data from outside the system (file, web service, etc) and turn it into a standard stream. In: files, external streams, Out: internal read stream.
    \item [Read Persistence] - Store reads on disk, index. In: read stream, Out: persistence confirmation events.
    \item [Read Statistics] - Look at read stream and calculate various approximate stats of interest - insert size, GC-bias,  etc. In: read stream, Out: running stats of interest
    \item [QC] - Compute QC score for reads. In: reads, Out: reads with QC score
    \item [Read Filtering] - Filter out low quality reads based on configured parameters. In: reads with QC score, Out: filtered reads.
    \item [Read Mapping] - Align reads to reference genome. In: stream of reads, Out: streams of mapped, unmapped, split reads.
    \item [Local Assembly] - Local assembly of reads into candidate haplotypes. In: stream of aligned reads, Out: Updated haplotypes event.
    \item [Haplotype Persistence] - Storage and lookup of candidate haplotypes. In: stream of reads, Out: persistence confirmation events, stream of haplotypes.
    \item [Variant Calling] - Evaluate candidate haplotypes for presence of variation. In: haplotype update events, Out: variant update events.
    \item [Variant Persistence] - Storage and lookup of variants. In: stream of reads, Out: persistence confirmation events, stream of variants.
    \item [Genotyping] - Genotype variant sites. In: variant update stream, Out: genotype update events.
    \item [Variant Filtering] - Filter out low quality variants. In: stream of variants, Out: filtered stream of variants.
    \item [Variant annotation] - Annotate variants for functional impact. In: stream of variants, Out: stream of annotated variants.
    \item [Output variants] - Format variants for external output. In: stream of variants, Out: files, external variant stream.
    \item [Notification] - Notify the user when events of interest occur. In: any stream, Out: stream of notifications.
\end{description}

\section{Proof of concept implementation}

Describe actual implementation efforts. Focus on very basic use case (take already mapped reads from a file, turn them into a stream, use stream to call SNPs, maybe some Indels/SVs). Demonstrate some comparison metrics compared to other callers. Demonstrate some service-level metrics (throughput etc.)

\section{Conclusions}
Rheos is great!